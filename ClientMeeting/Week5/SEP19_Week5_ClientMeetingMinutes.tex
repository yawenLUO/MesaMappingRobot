\documentclass[a4paper]{article}
\title{Minutes of Meeting}
\author{Group PG 19}
\date{25 August, 2015}

\begin{document}
\maketitle
\section*{Attendance}
\begin{tabular}{l l}
\textbf{Chair:} 
&a1690773 Wang Yuzhu\\
\textbf{Secretary:}&a1446944 Shaun Zander\\
&a1653772 Zhang Yun\\
&a1657343 Luo Yawen\\
&a1662541 Yang Jiajun\\
&a1653772 Zhang Yun\\
&a1671836 Wei Jingwen\\
\textbf{Absent:}&a1651541 Lee So Min (no apologies)
\end{tabular}

\section{Poster Presentation}
A Presentation ouf the poser was given with each group member formally introducing themselves. There was also an outline to the reasons behind the poster concept.

\section{GUI Mock-up Demonstration}
We provided paper dafts of the GUI to \textit{MesaMap} Clients for consideration.

\subsection{Start Screen}
\begin{itemize}
	\item The start screen \textbf{will not} require a confirmation dialogue before going to the main screen.
	\item The start screen \textbf{will} require a confirmation dialogue before exiting the program.
\end{itemize}

\subsection{Main Screen}
\begin{itemize}
	\item There will be X and Y coordinates for the map. However they will not be latitude and longitude.
	\item The battery indicator is fine as it is. There is no need to display the battery percentage.
	\item The location of the base may be further in to the centre of the map.
	\item There should be an "Emergency Stop" button to stop the robot. The suggestion is to change the "Energy" button to the Emergency Stop button (labelled as "Stop"). Could also toggle the button to be a "Start" button when the system has stopped.
	\item the "Help" and "Exit" button will also need to be moved to a different area out of the way of the main controls.
\end{itemize}

\subsection{NGZ Set Screen}
\begin{itemize}
	\item No Go Zones should be made by locating points on the map and then drawing a shape around those points.
	\item The No Go Zones should be all the same colour so as to be consistent to the user. 
\end{itemize}

\subsection{General}
The GUI is vary good on the whole. It will just a case of making minor changes and fine tuning the design.

\section{SVN Submission}
It will be easier if there was only one nominated person for submitting files to the SVN and doing web submissions. This is not a requirement but it would make it easier for the assessors.

\section{Next Meeting}
The next client meeting will be held at \textbf{3:30pm} in the \textbf{1st of September, 2015.} It will be held in the \textbf{SEP Lab}.

\end{document}